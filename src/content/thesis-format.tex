%----------------------------------------------------------------------------
\chapter{A dolgozat formai kivitele}
%----------------------------------------------------------------------------
Az itt található információk egy része a BME VIK Hallgatói Képviselet által készített ,,Utolsó félév a villanykaron'' c. munkából lett kis változtatásokkal átemelve. Az eredeti dokumentum az alábbi linken érhető el: \url{http://vik.hk/hirek/diplomafelev-howto-2015}.

%----------------------------------------------------------------------------
\section{A dolgozat kimérete}
%----------------------------------------------------------------------------
Szakdolgozat esetében minimum 30, 45 körüli ajánlott oldalszám lehet az iránymutató. De mindenképp érdemes rákérdezni a konzulensnél is az elvárásokra, mert tanszékenként változóak lehetnek az elvárások.

Mesterképzésen a Diplomatervezés 1 esetében a beszámoló még inkább az Önálló laboratóriumi beszámolókhoz hasonlít, tanszékenként eltérő formai követelményekkel, -- egy legalább 30 oldal körüli dolgozat az elvárt -- és az elmúlt fél éves munkáról szól. De egyben célszerű, ha ez a végleges diplomaterv alapja is. (A végleges 60-90 oldal körülbelül a hasznos részre nézve)


%----------------------------------------------------------------------------
\section{A dolgozat nyelve}
%----------------------------------------------------------------------------
Mivel Magyarországon a hivatalos nyelv a magyar, ezért alapértelmezésben magyarul kell megírni a dolgozatot. Aki külföldi posztgraduális képzésben akar részt venni, nemzetközi szintű tudományos kutatást szeretne végezni, vagy multinacionális cégnél akar elhelyezkedni, annak célszerű angolul megírnia diplomadolgozatát. Mielőtt a hallgató az angol nyelvű verzió mellett dönt, erősen ajánlott mérlegelni, hogy ez mennyi többletmunkát fog a hallgatónak jelenteni fogalmazás és nyelvhelyesség terén, valamint -- nem utolsó sorban -- hogy ez mennyi többletmunkát fog jelenteni a konzulens illetve bíráló számára. Egy nehezen olvasható, netalán érthetetlen szöveg teher minden játékos számára.

%----------------------------------------------------------------------------
\section{A dokumentum nyomdatechnikai kivitele}
%----------------------------------------------------------------------------
A dolgozatot A4-es fehér lapra nyomtatva, 2,5 centiméteres margóval (+1~cm kötésbeni), 11--12 pontos betűmérettel, talpas betűtípussal és másfeles sorközzel célszerű elkészíteni.

Annak érdekében, hogy a dolgozat külsőleg is igényes munka benyomását keltse, érdemes figyelni az alapvető tipográfiai szabályok betartására~\cite{Jeney}.
